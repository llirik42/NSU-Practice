\newcommand{\definition}[1]{\textbf{#1}}

\newcommand{\CLIA}{\textbf{CLI-A}}
\newcommand{\CLIB}{\textbf{CLI-B}}
\newcommand{\CLIEXT}{\textbf{CLI-EXT}}
\newcommand{\DC}{\textbf{DC}}
\newcommand{\UTM}{\textbf{UTM}}
\newcommand{\ISP}{\textbf{ISP}}

\begin{center}
	\textbf{\Large Введение}
\end{center}
\refstepcounter{chapter}
\addcontentsline{toc}{chapter}{Введение}

Цель прохождения практики состояла в развёртывании и анализе программно-аппаратного комплекса «РУБИКОН-К». Для достижения этой цели нужно было решить следующие задачи:
		
\begin{itemize}[nosep]
	\item изучить официальную документацию «Рубикон-К»;
			
	\item произвести инсталляция программного обеспечения, необходимого для развёртывания «Рубикон-К»;
			
	\item произвести настройку, тестирование и проверку работоспособности установленного «Рубикон-К»;
			
	\item Проанализировать и рассмотреть «Рубикон-К» в качестве межсетевого экрана.
\end{itemize}
		
\noindent Также дополнительной задачей являлось сравнение «Рубикон-К» с межсетевым экраном нового поколения <<Usergate>>.

Актуальность темы межсетевых экранов и систем обнаружения вторжений существенна в современном мире технологий, так как постоянно открываются новые способы осуществления атак на компьютерные сети, когда как межсетевые экраны и системы обнаружения вторжений позволяют защищаться от них. 

Предполагаемые результаты прохождения данной практики описывает следующий список:

\begin{itemize}[nosep]
	\item навыки чтения официальных документаций;
		
	\item умение производить инсталляцию ПО, необходимого для развёртывания других приложений;
		
	\item навыки настройки, тестирования и проверки работоспособности ПО;
	
	\item навыки в сравнении различных программно-аппаратных комплексов и межсетевых экранов.
\end{itemize}

\newpage

\begin{center}
	\textbf{\Large 1. Предметная область}
\end{center}
\refstepcounter{chapter}
\addcontentsline{toc}{chapter}{1. Предметная область}

\definition{Межсетевой экран} --- это локальное (однокомпонентное) или функционально - распределенное программное (программно-аппаратное) средство (комплекс), реализующее контроль за информацией, поступающей в автоматизированную систему и/или выходящей из автоматизированной системы \cite{fw}. Например, <<брандмауер>> в семействе операционных систем Windows можно считать межсетевым экраном. В общем случае межсетевые экраны позволяют лишь осуществлять контроль и фильтрацию трафика для защиты от различных атак (Например, IP-spoofing, SYN-flood).
		
\definition{Межсетевой экран нового поколения} (\textbf{NGFW} --- \textbf{N}ext \textbf{G}eneration \\\textbf{F}ire\textbf{w}all) --- межсетевой экран для глубокой фильтрации трафика, интегрированный с IDS (Intrusion Detection System, система обнаружения вторжений) или IPS (Intrusion Prevention System, система предотвращения вторжений) и обладающий возможностью контролировать и блокировать трафик на уровне приложений \cite{ngfw1}.

\definition{Межсетевой экран нового поколения} --- это комплексный инструмент, предназначенный для контроля трафика, управления доступом пользователей и приложений, предотвращения атак \cite{ngfw2}.

То есть межсетевой экран следующего поколения расширяет возможности <<обычного>> межсетевого экрана и объединяет в себе функциональность антивирусов, брандмауэров и других приложений безопасности.

\definition{Система обнаружения вторжений} (\textbf{IDS} --- \textbf{I}ntrusion \textbf{D}etection \textbf{S}ystem) --- специализированная система, используемая для идентификации того факта, что была предпринята попытка вторжения, вторжение происходит или произошло, а также для возможного реагирования на вторжение в информационные системы и сети \cite{ids}.

\definition{Система предотвращения вторжений} (\textbf{IPS} --- \textbf{I}ntrusion \textbf{P}revention \textbf{S}ystem) --- вид систем обнаружения вторжений, специально предназначенных для обеспечения активной возможности реагирования \cite{ips}.

\newpage

\begin{center}
	\textbf{\Large 2. <<Рубикон-К>>}
\end{center}
\refstepcounter{chapter}
\addcontentsline{toc}{chapter}{2. <<Рубикон-К>>}

Программно-аппаратный комплекс <<Рубикон-К>>, разработанный в компании <<Эшелон>>, объединяет функции маршрутизатора, межсетевого экрана типа «А» и типа «Б» четвертого класса защиты и системы обнаружения вторжений уровня сети четвертого класса защиты \cite{rubicon}. Комплекс сертифицирован ФСТЭК России. «Рубикон-К» имеет 4 варианта исполнения \cite{rubicon}:
\begin{enumerate}
	\item <<РУБИКОН-K mini>> --- для небольших сетей;
	
	\item <<РУБИКОН-K 1U>> --- для средних сетей;
		
	\item <<РУБИКОН-K Высокопроизводительный>> --- для больших сетей;
			
	\item <<Рубикон-К miniРУБИКОН-K Мультипортовый>> --- для крупных сетей.
\end{enumerate}
	
\noindent «Рубикон-К» обладает следующими преимуществами \cite{rubicon}:

\begin{itemize}
	\item web-интерфейс управления с ролевой моделью доступа;
	
	\item выполнение основных функций коммутации сетевых пакетов (коммутатор уровня L2 и коммутатор уровня L3);
	
	\item поддержка статической и динамической маршрутизации;
	
	\item возможность резервирования на уровне устройств (по протоколу CARP);
	
	\item возможность резервирования на уровне портов (bridge, VLAN, bonding);
	
	\item возможность резервирования на уровне каналов связи по средствам динамической маршрутизации с использованием протоколов OSPF, BGP;
	
	\item возможность построение VPN туннелей с использованием протоколов IPSec, OpenVPN и GRE;
	
	\item возможность трансляции сетевых адресов (NAT);
	
	\item выполнение фильтрации сетевых пакетов в режиме маршрутизатора (при использовании в режиме L3 коммутатора) по основным заголовкам сетевых пакетов;
	
	\item выполнение фильтрации сетевых пакетов в прозрачном режиме (при использовании в режиме L2 коммутатора) по основным заголовкам сетевых пакетов;
	
	\item возможность фильтрации сетевых пакетов по мандатным меткам отечественных защищенных операционных систем (Astra Linux и МСВС);
	
	\item наличие системы обнаружения вторжений (IDS);
	
	\item наличие системы предотвращения вторжений (IPS);
	
	\item возможность анализа сетевого трафика средствами СОВ, поступающего от внешних источников, с использованием технологии SPAN-порта;
	
	\item возможность функционирования СОВ в прозрачном режиме;
	
	\item наличие HTTP-прокси и FTP-прокси;
	
	\item возможность совместного использования HTTP-прокси с внешним антивирусом (по протоколу ICAP);
\end{itemize}

\begin{center}
	\textbf{\Large 2.1 Развёртывание}
\end{center}
\refstepcounter{section}
\addcontentsline{toc}{section}{2.1 Развёртывание}

Для прохождения учебной практики <<Рубикон-К>> был предоставлен в виде ISO-образа с некоторым дистрибутивом Linux, поэтому встал вопрос соответствующего его развёртывания. Для этого была выбрана программа типа Hypervisor <<Oracle VM VirtualBox>>, позволяющая производить инсталляцию, настройку и использование операционных систем, установленных на ISO-образах. Данное ПО было выбрано потому что на момент начала прохождения практики имелся положительный опыт при работе с ним.

\begin{center}
	\textbf{\Large 2.2 Настройка и тестирование}
\end{center}
\refstepcounter{section}
\addcontentsline{toc}{section}{2.2 Настройка и тестирование}

Для проверки работоспособности <<Рубикон-К>> в <<Oracle VM VirtualBox>> были настроены ещё три виртуальные машины, две из которых использовали операционную систему <<Windows 10-22h2>>, а оставшаяся --- <<Linux Mint 21.3 cinnamon>>.

Базовая настройка <<Рубикон-К>> производилась согласно руководству администратора \cite{doc}. Однако в ходе тестирования обнаружилось, что некоторые заявленные преимущества <<Рубикон-К>> в данной комплектации на самом деле не имеют места. Так, например, возможность трансляции сетевых адрес отсутствует \footnote{в руководстве администратора \cite{doc} сказано, что трансляция сетевых адресов происходит автоматически, а ручная настройка трансляции не предусмотрена.}. Из этого сразу же следует, что проверить возможности работы <<Рубикон-К>>, связанные с доступом к сети <<Интернет>> можно лишь частично \footnote{виртуальная машина с <<Рубикон-К>> имела доступ к сети <<Интернет>>, однако, для полной проверки её работоспособности, все остальные виртуальные машины в стенде должны были бы иметь доступ к сети <<Интернет>> исключительно через виртуальную машину с <<Рубикон-К>>, однако это не представляется возможным из-за отсутствия трансляции сетевых адресов в данной комплектации.}.

Несмотря на отсутствие трансляции сетевых адресов была проверена работоспособность следующих возможностей:
\begin{itemize}
	\item веб-интерфейс и его ролевая система;
	
	\item проверка статуса <<Рубикон-К>>;
	
	\item настройка сетевых интерфейсов;
	
	\item настройка меню веб-интерфейса;
	
	\item настройка статических и динамических маршрутов;
	
	\item DHCP-сервер;
	
	\item ограничения трафика.
\end{itemize}

\newpage

\begin{center}
	\textbf{\Large 3. Сравнение <<Рубикон-К>> и <<Usergate>>}
\end{center}
\refstepcounter{chapter}
\addcontentsline{toc}{chapter}{3. Сравнение <<Рубикон-К>> и <<Usergate>>}

<<Рубикон-К>> и <<Usergate>> являются современными межсетевыми экранами, но второй заявлен как межсетевой экран нового поколения, в то время как первый --- нет. Тем не менее, существуют различия в их возможностях:
\begin{itemize}
	\item <<Usergate>> поддерживает кластеризацию и отказоустойчивость, <<Рубикон-К>> --- нет \footnote{<<Рубикон-К>> поддерживает лишь резервное копирование и автоматическое восстановление, в то время как <<Usergate> поддерживает работу с кластером отказоустойчивости, работающем по протоколу VRRP.};
	
	\item <<Usergate>> позволяет объединять сетевые интерфейсы в группы, называемые <<зонами>>, для более быстрого и удобного их конфигурирования, поскольку доступна конфигурация как самих интерфейсов, так и <<зон>>;
	
	\item  <<Usergate>> позволяет объединять в общие группы номера телефонов, электронные адреса, сетевые интерфейсы, списки URL и приложения для более удобного их конфигурирования и использования. Например, можно запрещать или разрешать трафик через группу URL, а не через единичные URL;
	
	\item <<Usergate>> в отличие от <<Рубикон-К>> поддерживает конфигурацию всех типов трансляции сетевых адресов;

	\item <<Usergate>> поддерживает авторизацию пользователей, используя данные о записях из таких источников как <<LDAP>> и <<Active Directory>>, а также позволяет авторизовать пользователя используя прозрачную авторизацию по протоколу Kerberos, в то время как учётные записи в <<Рубикон-К>> настраиваются исключительно в нём, и при этом авторизоваться можно лишь по логину и паролю.
\end{itemize}

\newpage

\begin{center}
	\textbf{\Large Заключение}
\end{center}
\refstepcounter{chapter}
\addcontentsline{toc}{chapter}{Заключение}

В результате прохождения практики был развёрнут, проанализирован и оттестирован программно-аппаратный комплекс <<Рубикон-К>>. Также было проведено его сравнение с межсетевым экраном нового поколения <<Usergate>>, показавшее, что в <<Usergate>> больше возможностей, чем в <<Рубикон-К>>. Также был получен опыт изучения официальных документаций.

В течение прохождения практики пришлось столкнуться с двумя основными трудностями:
\begin{enumerate}
	\item ISO-образ, в виде которого поставляется <<Рубикон-К>>, изначально имел пароль, не указанный в документации, поэтому пришлось найти способ сброса пароля без повреждения ISO-образа;
	
	\item Как уже было упомянуто ранее, использованная комплектация <<Рубикон-К>> не поддерживает трансляцию сетевых адресов, что ограничивает возможности его использования и тестирования.
\end{enumerate}

\printbibliography[heading=bibintoc, title={Список использованных источников}]
