\newcommand{\definition}[1]{\textbf{#1}}

\newcommand{\CLIA}{\textbf{CLI-A}}
\newcommand{\CLIB}{\textbf{CLI-B}}
\newcommand{\CLIEXT}{\textbf{CLI-EXT}}
\newcommand{\DC}{\textbf{DC}}
\newcommand{\UTM}{\textbf{UTM}}
\newcommand{\ISP}{\textbf{ISP}}

\begin{center}
	\textbf{\Large 1. Введение}
\end{center}
\refstepcounter{section}
\addcontentsline{toc}{section}{1. Введение}

Цель прохождения практики состояла в изучение межсетевых экранов следующего поколения «UserGate» в условиях реальной эксплуатации. Для достижения этой цели нужно было решить следующие задачи:
		
\begin{itemize}[nosep]
	\item Изучить основы информационной безопасности, компьютерных сетей и основных интернет-протоколов;
			
	\item Изучить принципы работы межсетевого экрана нового поколения <<Usergate>>;
			
	\item Научиться создавать «стенды» в гипервизоре Oracle VM VirtualBox для эмуляции работы межсетевого экрана;
			
	\item Научиться использовать <<стенды>> для решения проблем, возникших у клиентов, использующих <<Usergate>>.
\end{itemize}
		
Актуальность темы межсетевых экранов следующего поколения существенна в современном мире технологий, так как постоянно открываются новые способы осуществления атак на компьютерные сети, в то время как межсетевые экраны позволяют защищаться от них. Межсетевые экраны следующего поколения же являются их <<продолжением>> и поэтому позволяют гораздо более эффективно защищаться от различных атак. Также помимо этого они позволяют настраивать инфраструктуру и топологию компьютерной сети, производить мониторинг и диагностику, что делает их инструментом не только для предотвращения атак, но и для администрирования компьютерных сетей.
		
Организация, Usergate, в которой проходила практика, является российским разработчиком программного обеспечения и микроэлектроники и обеспечивает своими решениями информационную безопасность корпоративных сетей самого разного размера от малого и среднего бизнеса до крупных корпораций с распределенной инфраструктурой. В группу компаний UserGate входят ООО «Юзергейт» (разработка программного обеспечения) и ООО «Катунь Электроника» (разработка в области микроэлектроники) \cite{usergate}.		
		
Практика проходила в подразделении под названием <<Tech support>>, которое занимается решением проблем, возникших у клиентов Usergate, посредством создания <<стендов>> и эмуляции возникших сбоев и ошибок.
		
Предполагаемые результатами прохождения данной практики описывает следующий список:

\begin{itemize}[nosep]
	\item Понимание основ информационной безопасности, компьютерных сетей и основных интернет-протоколов;
				
	\item Умение пользоваться современными гипервизорами (в том числе Oracle VM VirtualBox);
			
	\item Умение администрировать межсетевые экраны следующего поколения\\<<Usergate>>;
			
	\item Умение эмулировать ошибки и сбои в компьютерных сетях посредством использование <<стендов>>.
\end{itemize}

\begin{center}
	\textbf{\Large 2. Предметная область}
\end{center}
\refstepcounter{section}
\addcontentsline{toc}{section}{2. Предметная область}

\definition{Межсетевой экран} --- это это программный или программно-аппаратный комплекс, предназначенный для контроля и фильтрации проходящего через него сетевого трафика в соответствии с заданными правилами. Например, <<брандмауер>> в семействе операционных систем Windows можно считать межсетевым экраном. В общем случае межсетевые экраны позволяют лишь осуществлять контроль и фильтрацию трафика для защиты от различных атак (Например, IP-spoofing, SYN-flood).
		
В то время как \definition{межсетевой экран нового поколения} (\textbf{NGFW} --- \textbf{N}ext \\\textbf{G}eneration \textbf{F}ire\textbf{w}all) --- это комплексный инструмент, предназначенный для контроля трафика, управления доступом пользователей и приложений, предотвращения атак. Межсетевой экран следующего поколения объединяет в себе функционал антивирусов, брандмауэров и других приложений безопасности. То есть межсетевой экран нового поколения совмещает в себе все возможности межсетевого экрана, но к этому ещё добавляет новый функционал. Также негласным правилом является то что межсетевой экран нового поколения анализирует трафик на всех уровнях модели OSI, в то время как обычный межсетевой экран --- только до четвёртого уровня включительно.

\begin{center}
	\textbf{\Large 3. Принципы работы <<Usergate>>}
\end{center}
\refstepcounter{section}
\addcontentsline{toc}{section}{3. Принципы работы <<Usergate>>}

<<Usergate>> является межсетевым экраном \footnote{Компания Usergate предоставляет как аппаратные, так и программные решения, однако в рамках практики работа происходила лишь с программным решением --- виртуальным межсетевым экраном нового поколения.} со следующими возможностями \cite{usergate_capabilities}:
		
\begin{itemize}
	\item Системы обнаружения вторжений (IDS/IPS);
	\item Advanced Threat Protection \footnote{Advanced Threat Protection включает в себя фильтрацию по категориям URL Filtering 3.0, морфологический анализ контента, облачный антивирус, поддержку списков Роскомнадзора, модуль блокировки рекламы ADblock.} (Опция);
	\item Доступ к внутренним ресурсам через SSL VPN Portal;
	\item Анализ и выгрузка информации об инцидентах безопасности (SIEM);
	\item Обратный прокси (Reverse proxy);
	\item Автоматизация реакции на угрозы безопасности информации (SOAR);
	\item Контроль Приложений L7;
	\item Защита почты (Mail Security) \footnote{Mail Security включает в себя антиспам, антивирусную проверку почты, поддержку методов фильтрации нежелательной почты.} (Опция);
	\item Контроль доступа в интернет;
	\item Дешифрование SSL;
	\item Гостевой портал;
	\item Идентификация пользователей;
	\item Виртуальная частная сеть (VPN);
	\item Поддержка АСУ ТП (SCADA);
	\item Удаленное администрирование;
	\item Безопасная публикация внутренних ресурсов и сервисов;
	\item Поддержка концепции BYOD (Bring Your Own Device);
	\item Поддержка высокой отказоустойчивости и кластеризации.
\end{itemize}
	
То есть <<Usergate>> не только производит анализ и фильтрацию сетевого трафика, но ещё и позволяет организовывать инфраструктуру сети (позволяет настройки правил NAT, выступать в качестве DHCP-сервера и т.д.), производить удалённое администрирование, мониторинг и диагностику трафика, управлять авторизацией пользователей, их правами и ролями и т.д.

\begin{center}
	\textbf{\Large 4. Создание и использование <<стендов>>}
\end{center}
\refstepcounter{section}
\addcontentsline{toc}{section}{4. Создание и использование <<стендов>>}

<<Usergate>> как виртуальный межсетевой экран нового поколения представлен фактически операционной системой, которая может установлена из ISO-образа. ISO-образ можно импортировать в любом современном гипервизоре (например, OracleVM VirtualBox или VMware Workstation). В рамках практики использовался гипервизор OracleVM VirtualBox.
		
Для эмуляции работы компьютерной сети необходимо эмулировать всех хостов в сети. В рамках обучения стенд состоял из шести виртуальных машин:
		
\begin{itemize}
	\item Виртуальная машина с <<Usergate>> (\UTM{} --- \textbf{U}nified \textbf{T}hreat \textbf{M}anagement);
			
	\item Виртуальная машина на операционной системе Debian 12.4.0, эмулирующая интернет-провайдера (\ISP{} --- \textbf{I}nternet \textbf{S}ervice \textbf{P}rovider);
			
	\item Виртуальная машина на операционной системе Windows Server 2012, эмулирующая сервер некоторой компании, находящийся в локальной сети компании (\DC{} --- \textbf{D}ata \textbf{C}enter).
			
	\item Три виртуальные машины на операционной системе Windows-10, две из которых эмулируют клиентов, находящихся в локальной сети некоторой компании (\CLIA{} и \CLIB{}, \textbf{CLI} --- \textbf{Cli}ent), а третий --- клиента вне локальной сети компании, но имеющего доступ к локальной сети с помощью Remote Access VPN (\CLIEXT{} --- \textbf{Cli}ent \textbf{Ext}ernal).
\end{itemize}
		
Стоит отметить, что конфигурация <<стенда>> заключается не только в поиске и импорте в гипервизор нужных ISO-образов операционных систем. Для полноценной работы стенда необходимо: осуществить настройку сетевой топологии в самом гипервизоре (указать сколько каждая виртуальная машины будет иметь виртуальных сетевых интерфейсов и какого они будут типа), осуществить настройку каждой виртуальной машины по-отдельности. 
		
\CLIA{}, \CLIB{} и \DC{} имели лишь один виртуальный сетевой интерфейс каждая, при этом интерфейсы этих виртуальных машин находились в одной локальной сети (\textit{LOCAL-NETWORK}) \footnote{Они находились в виртуальной локальной сети гипервизора, но тем не менее они не имели прямого доступа к сетевому пространству операционной системы, на которой был запущен гипервизор.}. \UTM{} же имела 2 виртуальных интерфейса: один --- в виртуальной локальной сети \textit{LOCAL-NETWORK}, а второй --- в виртуальной локальной сети \textit{ISP-NETWORK}. \ISP{} имела 2 виртуальных интерфейса: один в виртуальной локальной сети \textit{ISP-NETWORK}, а второй виртуальный интерфейс имел тип <<bridge>>, что означает, что он имеет доступ к сетевому пространству <<хостовой>> операционной системы. \CLIEXT{} имела лишь один интерфейс, у которого был тип <<bridge>>.
			
На \CLIA{} и \CLIB{} не было никаких настроек поскольку все нужны IP-адреса и доступ к интернету они получали от \UTM{}. На \DC{} нужно было установить и настроить необходимые сервисы (такие как <<Active Directory>>, <<DNS-Manager>>, центр управления сертификатами и т.д.), а все нужные IP-адреса и доступ в интернет виртуальная машина опять-таки получала от \UTM{}. На \ISP{} были настроены правила NAT и статические маршруты для доступа в интернет. На \CLIEXT{} был настроен доступ к \UTM{} с помощью VPN (указан IP-адрес нужного интерфейса \UTM{} и тип подключения). На \UTM{} были произведены самые различные настройки:
		
\begin{itemize}
	\item Настроена фильтрация контента по протоколам, URL, IP-адресам клиентов;
			
	\item Настроены captive-портал и способы авторизации (через captive-портал, <<прозрачная>> авторизация по протоколу Kerberos);
			
	\item Настроен сбор диагностических данных;
			
	\item Настроены сценарии при атаках;
			
	\item Настроена проверка на писем электронной почты на наличие <<вирусов>> с помощью протокола ICAP.
\end{itemize}
		
Данный <<стенд>> использовался в рамках обучения. После обучения создавались различные <<стенды>> для решения проблем, возникших у пользователей продуктов Usergate.

\begin{center}
	\textbf{\Large 5. Заключение}
\end{center}
\refstepcounter{section}
\addcontentsline{toc}{section}{5. Заключение}

В результате прохождения практики были изучены принципы работы межсетевого экрана следующего поколения <<Usergate>> и способы его администрирования. Также были получены навыки в создании <<стендов>> в гипервизоре OracleVM VirtualBox для эмуляции работы компьютерных сетей и межсетевых экранов.
		
Также был получен опыт работы с литературными источниками и выделения необходимой и полезной информации, а также опыт работы в крупной российской компании наряду с её штатными сотрудниками.
		
В течение прохождения практики пришлось столкнуться с двумя трудностями: трудность в работе с гипервизором OracleVM VirtualBox (а именно, настройке сетевой топологии); трудность в понимании принципов работы <<Usergate>> ввиду многоплановости и сложности данного продукта.

\printbibliography[heading=bibintoc, title={Список литературы}]
